\documentclass[a4paper, polish, 14pt]{extarticle}
\usepackage{polski}
\usepackage[utf8]{inputenc}
\usepackage{enumitem, amssymb}
\usepackage{pifont}
\usepackage{hyphenat}
\usepackage{extsizes}
\newcommand{\xmark}{\ding{55}}
\pagestyle{empty}

\newlist{terms}{itemize}{2}
\setlist[terms]{label=$\square$}

\date{\vspace{-5ex}}
\setlength{\parindent}{0em}
\setlength{\parskip}{1em}




\begin{document}
\begin{center}
				\textsc{\Large Informacja dla uczestników badania}
\end{center}

Zapraszamy do udziału w badaniu dotyczącym kształtowania się intuicji filozoficznych. Chcemy zbadać, czy studia wpływają na to, jak myślimy o pewnych hipotetycznych sytuacjach, często bardzo oderwanych od rzeczywistości.

Chcielibyśmy prosić o udział w badaniu, które będzie trwało przez cały czas Państwa studiów (2 lub 3 lata). Raz na semestr będziemy prosić, by przeczytali Państwo kilka hipotetycznych scenariuszy i odpowiedzieli na dotyczące ich pytania. Pozwoli nam to zobaczyć, czy mają Państwo stałe intuicje, czy też w trakcie studiów z upływem czasu się one u Państwa zmieniają.

Badanie jest anonimowe, udział w każdym z etapów nie zajmie Państwu więcej niż 15 minut. W badaniu nie ma ,,dobrych'' ,,poprawnych'' ,,złych'' czy ,,błędnych'' odpowiedzi. Chcemy dowiedzieć się, jakie mają Państwo intuicje na temat czytanych historii. 

Ankiety będą Państwo oznaczać specjalnymi kodami tak, aby nikt z~analizujących odpowiedzi nie mógł stwierdzić, że są to Państwa odpowiedzi, ale też tak, byśmy mogli prześledzić odpowiedzi pojedynczych anonimowych osób w kolejnych latach badania. Państwa odpowiedzi nie będą w żaden sposób wpływały na Państwa oceny na studiach.

Osoby, które wyrażą zgodę, będą proszone o przekazanie prowadzącym badania swojego adresu mailowego, byśmy mogli w przyszłości zapraszać Państwa do udziału w kolejnych jego etapach. Z udziału w badaniu można zrezygnować na każdym etapie, nie będą się z tym wiązać żadne konsekwencje.

Serdecznie zapraszamy do udziału w badaniu! Wszelkie pytania i wątpliwości prosimy kierować do Katarzyny Kuś (Laboratorium Filozofii Eksperymentalnej): \textsf{kkus@uw.edu.pl}

\end{document}
