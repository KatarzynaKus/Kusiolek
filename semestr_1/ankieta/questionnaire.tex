\documentclass[
  polish,
  pagemark,
	print_questionnaire_id,
	oneside,
  stamp]{sdapsclassic}
\usepackage[utf8]{inputenc}
\usepackage{multicol}
\usepackage{pgfkeys}
\usepackage[printwatermark]{xwatermark}

% Logo

\newwatermark[firstpage,picfile = {kognilablogo},picfileext=png, picscale=2.5, picypos = 164, picxpos = -50]{}

% To jest plik ze zmiennymi. Używamy ich zamiast tekstu.
% W tym pliku znajdują się zmienne, których będziemy używać w ankiecie.
% Są wyrzucone do osobnego pliku z szablonu ankiety.

% Zmienne definiujemy za pomocą:
%\pgfkeyssetvalue{nazwa_zmiennej}{tekst}
% W pliku `questionnaire.tex` odwołujemy się do nich przez:
%\gv{nazwa_zmiennej}


% Ogólne elementy tekstowe ankiety

\pgfkeyssetvalue{survey_authors}{
Pracownia filozofii eksperymentalnej "KogniLab"
}

\pgfkeyssetvalue{survey_title}{
Kształtowanie się intuicji filozoficznych
}

\pgfkeyssetvalue{survey_info}{
Tutaj znajdą się informacje o badaniu i inne informacje niezbędne respondentom.
}

% Elementy konstrukcyjne pytań
\pgfkeyssetvalue{yes_a}{ % Odpowiedź "Tak"
Tak
}

\pgfkeyssetvalue{no_a}{ % Odpowiedź "Nie"
Nie
}

\pgfkeyssetvalue{left_anchor}{ % Lewa strona dyferencjału semantycznego
słabo
}

\pgfkeyssetvalue{right_anchor}{ % Prawa strona dyferencjału semantycznego
mocno
}

\pgfkeyssetvalue{yesno_question}{ % Formuła pytania rozstrzygnięcia
Czy zgadzasz się ze stwierdzeniem
}

\pgfkeyssetvalue{level_question}{ % Formuła pytania o siłę przekonania
Jak bardzo jesteś pewny swojej odpowiedzi?
}

\pgfkeyssetvalue{male}{ 
Mężczyzna
}

\pgfkeyssetvalue{female}{
Kobieta
}

\pgfkeyssetvalue{other}{
Inna
}

\pgfkeyssetvalue{refuse}{
Nie chcę odpowiadać na to pytanie.
}

% Sekcja z danymi demograficznymi badanych

\pgfkeyssetvalue{demography_section}{
Dane demograficzne
}

\pgfkeyssetvalue{main_field_q}{
Podstawowy kierunek studiów
}

\pgfkeyssetvalue{birth_year_q}{
Rok urodzenia
}

\pgfkeyssetvalue{gender_q}{
Płeć
}

\pgfkeyssetvalue{philosophy_hs_q}{
Czy w liceum miałeś zajęcia z filozofii
}

\pgfkeyssetvalue{olympiad_q}{
Czy startowałeś kiedyś w Olimpiadzie Filozoficznej lub Konkursie Filozoficznym?
}

\pgfkeyssetvalue{philosophy_start_year_q}{
W którym roku rozpocząłeś studia filozoficzne? \\ \textit{Jeśli tak, to jakie i na którym roku?}
}

\pgfkeyssetvalue{other_fields_q}{
Czy w ubiegłych latach podejmowałeś już studia na innym kierunku niż filozofia?
}


% Sekcja z pierwszym scenariuszem badawczym

\pgfkeyssetvalue{story_1_section}{
Historyjka 1
}

\pgfkeyssetvalue{story_1_text}{
Tutaj będzie tekst pierwszej historyjki
}

\pgfkeyssetvalue{story_1_q1}{
Stwierdzenie jeden
}


% Tutaj redefiniuję komendę do wywoływania zmiennych, aby łatwiej było ją wpisywać.
% Od tej pory aby użyć wartości zmiennej można posługiwać się poleceniem \gv{nazwa_zmiennej}
\newcommand{\gv}{\pgfkeysvalueof}

% Autorzy ankiety
\author{\gv{survey_authors}}
% Tytuł ankiety
\title{\gv{survey_title}}
\setlength{\columnseprule}{0.5pt}

\begin{document}
% Cała ankieta musi znaleźć się w środowisku `questionnaire`
  \begin{questionnaire}
% To jest komenda dodająca informacje o ankiecie na górze pierwszej strony.
    \begin{info}
						\gv{survey_info}
    \end{info}

    \addinfo{Date}{10.03.2013}

		\section{\gv{demography_section}}% Tak dodajemy nagłówki sekcji

		\begin{multicols}{2}

	  \textbox*{3cm}{\gv{sticker}}

		\textbox*{1.2cm}{\gv{main_field_q}}% Tak dodajemy pytania otwarte

		\textbox*{1.2cm}{\gv{birth_year_q}}


		\begin{choicequestion}[cols=2]{\gv{gender_q}}% Tak dodajemy pytania zamknięte
						\choiceitem{\gv{female}}
						\choiceitem{\gv{male}}
						\choiceitem{\gv{other}}
						\choiceitem{\gv{refuse}}
    \end{choicequestion}


		\begin{choicequestion}[cols=2]{\gv{philosophy_hs_q}}
      \choiceitem{\gv{yes_a}}
      \choiceitem{\gv{no_a}}
    \end{choicequestion}

		\begin{choicequestion}[cols=2]{\gv{olympiad_q}}
      \choiceitem{\gv{yes_a}}
      \choiceitem{\gv{no_a}}
    \end{choicequestion}

		\textbox*{1.2cm}{\gv{philosophy_start_year_q}}


		\begin{choicequestion}[cols=2]{\gv{other_fields_q}}
      \choiceitem{\gv{yes_a}}
      \choiceitem{\gv{no_a}}


			\choiceitemtext{1.2cm}{2}{a.}
			\choiceitemtext{1.2cm}{2}{b.}
			\choiceitemtext{1.2cm}{2}{c.}
    \end{choicequestion}

		\begin{choicequestion}[cols=2]{\gv{philosophy_fields_q}}
						\choiceitem{\gv{epistemology}}
						\choiceitem{\gv{logic}}
						\choiceitem{\gv{ethics}}
						\choiceitem{\gv{polit_philosophy}}
						\choiceitem{\gv{soc_philosophy}}
						\choiceitem{\gv{hist_philosophy}}
						\choiceitem{\gv{cul_philosophy}}
						\choiceitem{\gv{aesthethics}}
						\choiceitem{\gv{rel_philosophy}}
						\choiceitem{\gv{sci_philosophy}}
						\choiceitem{\gv{anal_philosophy}}
						\choiceitem{\gv{ontology}}
						\choiceitem{\gv{cont_philosophy}}
						\choiceitemtext{1.2cm}{2}{Inne:}


   	\end{choicequestion}

    \end{multicols}

		\newpage

		\section{\gv{story_1_section}}

    \gv{story_1_text}

		\begin{choicequestion}[cols=2]{\gv{yesno_question} \\ \textbf{\gv{story_1_q1}}}
      \choiceitem{\gv{yes_a}}
      \choiceitem{\gv{no_a}}
		\end{choicequestion}

		\singlemark{\gv{level_question}}{\gv{left_anchor}}{\gv{right_anchor}}

  \end{questionnaire}
\end{document}

