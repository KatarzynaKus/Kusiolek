\pgfkeyssetvalue{story_2_section}{
Maszyna przyjemności
}

\pgfkeyssetvalue{story_2_text}{
Któregoś dnia słyszysz dzwonek do drzwi. W progu stoi wysoki mężczyzna w czarnym płaszczu i okularach przeciwsłonecznych. Przedstawia się jako Smith. Twierdzi, że ma dla ciebie bardzo ważną propozycję. Trochę zaniepokojony, ale ciekawy zapraszasz go do środka. „Zostałeś wskazany przez nasz system jako idealny kandydat” mówi Smith. „Możemy podłączyć twój mózg do stworzonej przez nasz zespół neuronaukowców maszyny symulującej doświadczenie. Podczas gdy twoje ciało będzie w maszynie, będziemy mogli stymulować twój mózg tak, by to, czego doświadczysz, nie różniło się jakościowo od myśli, przeżyć ani uczuć, które możesz mieć w świecie rzeczywistym. Możesz powiedzieć nam dokładnie, co chciałbyś przeżyć i osiągnąć i zamienić swoje życie w satysfakcjonujące pasmo przyjemności. My odpowiednio zaprogramujemy maszynę, byś miał odpowiednie wrażenia i przeżycia. Będziesz ze wszech miar szczęśliwy. W maszynie zapomnisz, że kiedykolwiek do niej wchodziłeś, wszystko będzie wydawało ci się rzeczywiste. Nie musisz martwić się też o rodzinę i bliskich – im też zaproponujemy wejście do maszyn, będą więc mogli zadecydować o swoim życiu i szczęściu”.
}

\pgfkeyssetvalue{story_2_q1}{
Co byś wybrał?
}

\pgfkeyssetvalue{story_2_q1a1}{
podłączyć się do maszyny
}

\pgfkeyssetvalue{story_2_q1a2}{
pozostać w świecie rzeczywistym
}

