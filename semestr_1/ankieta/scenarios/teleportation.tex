\pgfkeyssetvalue{teleportation_section}{
Teleportacja
}

\pgfkeyssetvalue{teleportation_text}{
Jest 2450 rok. Cywilizacja ludzka dokonała ledwo wyobrażalnego dla nas skoku technologicznego. Derek mieszka na Ziemi, jego żona jest na Wenus, a matka na Marsie. Derek, mając dość samotności, wchodzi do teleportera kwantowego, który ma na Ziemi blisko domu, i mówi: „Chcę spotkać się i z żoną, i z mamą”. Naciska guzik. Po chwili całe jego ciało jest już przeskanowane, a informacja o strukturze komórek i stanie umysłowym zostaje wysłana tunelem czasoprzestrzennym na Wenus i na Marsa, gdzie zostaje bezbłędnie odwzorowana w formie fizycznej. Po chwili na Wenus w apartamencie żony Dereka w teleporterze pojawia się postać. Kobieta obejmuje ją z radością w oczach, mówiąc: „Mój drogi, jak miło cię widzieć!”. Oboje są szczęśliwi. W tym samym czasie również na Marsie u matki Dereka z teleportera wychodzi postać. Kobieta obejmuje ją z radością w oczach, mówiąc: Mój drogi, jak miło cię widzieć!”. Dom Dereka na Ziemi jest teraz pusty.
}

\pgfkeyssetvalue{teleportation_q1}{
Która możliwość najlepiej opisuje to, co się stało?
}

\pgfkeyssetvalue{teleportation_q1a1}{
Derek objął swoją żonę, ale matkę objął ktoś inny.
}

\pgfkeyssetvalue{teleportation_q1a2}{
Derek objął swoją matkę, ale żonę objął ktoś inny.
}

\pgfkeyssetvalue{teleportation_q1a3}{
Ktoś inny objął żonę, ktoś inny matkę, żadnej nie obejmował Derek.
}

\pgfkeyssetvalue{teleportation_q1a4}{
Derek objął zarówno żonę, jak i matkę.
}
