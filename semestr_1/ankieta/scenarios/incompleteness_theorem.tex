\pgfkeyssetvalue{incompleteness_theorem_section}{
Niezupełność arytmetyki
}

\pgfkeyssetvalue{incompleteness_theorem_text}{
Wyobraźmy sobie, że Jan na studiach dowiedział się, że to Gödel udowodnił jedno z najważniejszych twierdzeń matematyki, tak zwane twierdzenie o niezupełności. Jan, będąc niezgorszym matematykiem, potrafi zrekonstruować dowód, a za jego autora uważa oczywiście Gödla. Nie wie jednak nic więcej o Gödlu. A teraz wyobraźmy sobie, że Gödel nie jest w rzeczywistości autorem tego twierdzenia. Jako pierwszy dowód przeprowadził człowiek o nazwisku „Schmidt”, którego ciało zostało znalezione w tajemniczych okolicznościach wiele lat temu w Wiedniu. Jego przyjaciel Gödel przywłaszczył sobie manuskrypt i przypisał sobie autorstwo dowodu. W ten sposób Gödel zapisał się na kartach historii jako twórca dowodu niezupełności arytmetyki. Większość ludzi, którzy kiedykolwiek słyszeli nazwisko ,,Gödel'' jest jak Jan: jedyne, co słyszeli o Gödlu, to to, że jako pierwszy przeprowadził dowód twierdzenia o niezupełności.
}

\pgfkeyssetvalue{incompleteness_theorem_q1}{
O kim mówi John, gdy używa nazwiska ,,Gödel''
}

\pgfkeyssetvalue{incompleteness_theorem_q1a1}{
O człowieku, który w rzeczywistości odkrył twierdzenie o niezupełności.
}

\pgfkeyssetvalue{incompleteness_theorem_q1a2}{
O człowieku, który ukradł manuskrypt i przypisał sobie jego autorstwo.
}
