% W tym pliku znajdują się zmienne, których będziemy używać w ankiecie.
% Są wyrzucone do osobnego pliku z szablonu ankiety.

% Zmienne definiujemy za pomocą:
%\pgfkeyssetvalue{nazwa_zmiennej}{tekst}
% W pliku `questionnaire.tex` odwołujemy się do nich przez:
%\gv{nazwa_zmiennej}


% Ogólne elementy tekstowe ankiety

\pgfkeyssetvalue{survey_authors}{
Pracownia filozofii eksperymentalnej "KogniLab"
}

\pgfkeyssetvalue{survey_title}{
Kształtowanie się intuicji filozoficznych
}

\pgfkeyssetvalue{survey_info}{
Tutaj znajdą się informacje o badaniu i inne informacje niezbędne respondentom.
}

\pgfkeyssetvalue{sticker}{
Proszę przykleić w tym miejscu naklejkę z kodem
}

% Elementy konstrukcyjne pytań
\pgfkeyssetvalue{yes_a}{ % Odpowiedź "Tak"
Tak
}

\pgfkeyssetvalue{no_a}{ % Odpowiedź "Nie"
Nie
}

\pgfkeyssetvalue{left_anchor}{ % Lewa strona dyferencjału semantycznego
słabo
}

\pgfkeyssetvalue{right_anchor}{ % Prawa strona dyferencjału semantycznego
mocno
}

\pgfkeyssetvalue{yesno_question}{ % Formuła pytania rozstrzygnięcia
Czy zgadzasz się ze stwierdzeniem
}

\pgfkeyssetvalue{level_question}{ % Formuła pytania o siłę przekonania
Jak bardzo jesteś pewny swojej odpowiedzi?
}

\pgfkeyssetvalue{male}{ 
Mężczyzna
}

\pgfkeyssetvalue{female}{
Kobieta
}

\pgfkeyssetvalue{other}{
Inna
}

\pgfkeyssetvalue{refuse}{
Nie chcę odpowiadać na to pytanie.
}

% Sekcja z danymi demograficznymi badanych

\pgfkeyssetvalue{demography_section}{
Dane demograficzne
}

\pgfkeyssetvalue{main_field_q}{
Podstawowy kierunek studiów
}

\pgfkeyssetvalue{birth_year_q}{
Rok urodzenia
}

\pgfkeyssetvalue{gender_q}{
Płeć
}

\pgfkeyssetvalue{philosophy_hs_q}{
Czy w liceum miałeś zajęcia z filozofii
}

\pgfkeyssetvalue{olympiad_q}{
Czy startowałeś kiedyś w Olimpiadzie Filozoficznej lub Konkursie Filozoficznym?
}

\pgfkeyssetvalue{philosophy_start_year_q}{
W którym roku rozpocząłeś studia filozoficzne?
}

\pgfkeyssetvalue{other_fields_q}{
Czy w ubiegłych latach podejmowałeś już studia na innym kierunku niż filozofia? \\ \textit{Jeśli tak, to jakie i na którym roku?}
}

\pgfkeyssetvalue{philosophy_fields_q}{
Które dziedziny filozofii interesują cię najbardziej?
}

\pgfkeyssetvalue{epistemology}{
Epistemologia
}

\pgfkeyssetvalue{ethics}{
Etyka
}

\pgfkeyssetvalue{polit_philosophy}{
Filozofia polityki
}

\pgfkeyssetvalue{soc_philosophy}{
Filozofia społeczna
}

\pgfkeyssetvalue{hist_philosophy}{
Historia filozofii
}

\pgfkeyssetvalue{logic}{
Logika
}

\pgfkeyssetvalue{aesthethics}{
Estetyka
}

\pgfkeyssetvalue{rel_philosophy}{
Filozofia religii
}

\pgfkeyssetvalue{sci_philosophy}{
Filozofia nauki
}

\pgfkeyssetvalue{anal_philosophy}{
Filozofia analityczna
}

\pgfkeyssetvalue{cul_philosophy}{
Filozofia kultury
}

\pgfkeyssetvalue{cont_philosophy}{
Filozofia kontynentalna
}
% Sekcja z pierwszym scenariuszem badawczym

\pgfkeyssetvalue{story_1_section}{
Historyjka 1
}

\pgfkeyssetvalue{story_1_text}{
Tutaj będzie tekst pierwszej historyjki
}

\pgfkeyssetvalue{story_1_q1}{
Stwierdzenie jeden
}
